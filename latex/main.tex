\documentclass[a4paper, twoside]{report}

\usepackage[english]{babel}
\usepackage[utf8x]{inputenc}
\usepackage[T1]{fontenc}
\usepackage{listings}
\usepackage{hyperref}
\hypersetup{colorlinks=false}
\usepackage{lscape}
\usepackage{subfigure}
\usepackage{amsmath}
\usepackage{graphicx}
\usepackage[colorinlistoftodos]{todonotes}
\newcommand{\subsubsubsection}[1]{\paragraph{#1}\mbox{}\\}
\setcounter{secnumdepth}{4}
\setcounter{tocdepth}{4}
%% Sets page size and margins
\usepackage[a4paper,top=3cm,bottom=2cm,left=3cm,right=3cm,marginparwidth=1.75cm]{geometry}
\providecommand{\keywords}[1]{\textbf{\textit{Index terms---}} #1}

\title{Upgrading the HEP-Frame Scheduling Dependency Graph To support Conditional Task Graphs }
\author{José Pedro Moreira Resende}

\begin{document}
\input{title/title.tex}

\begin{abstract}
Scientists often develop applications to analyse large datasets and process raw data into useful information with scientific value. These applications are typically structured as a sequence of tasks in a pipeline of actions, where data can be modified at each pipeline stage, filtered out and/or output as a result. The elements that are filtered out are not processed by the next tasks in the pipeline. Actions often vary from intensive computing tasks to simple evaluations that may discard irrelevant data.
\par The increasing complexity of the algorithms and the size of the datasets to be analyzed makes this type of problem computationally intensive and hard to end in a timely manner. Since these applications are usually developed by the end-user, a non-programming expert working in a scientific domain, such as physics or chemistry, the overall performance of the code execution may even be more affected. This is due to the fact that scientists focus their programming effort in the correctness of the algorithm implementation and the time to obtain results, leaving behind issues like the performance of the resulting code (and associated data structures) and its execution on specific hardware platforms.
\par Current compute servers are inherently very parallel, supporting several forms of parallel execution of code: from simultaneous execution of several tasks (on an increasing number of physical cores or processing units, PUs) to the execution of the same task (or sets of instructions) on different data elements (aka known as vector computing). A framework was deployed to aid scientists to develop scientific applications that are based on a pipeline of actions (that may be discarded along its path) applied to a very large dataset, and to efficiently manage their code execution in homogeneous and heterogeneous servers: the Highly Efficient Pipeline Framework, HEP-Frame.
\par The current HEP-Frame version only supports one type of decision at the end of each action or pipeline stage: either the outcome of the action satisfies a given criterium (or set of criteria) and follows the pipeline path, or fails and the data element is no longer processes, i.e., is discarded. However, some scientific applications, such as particle analysis, require that the decision on some pipeline stages may lead to different alternative solutions, namely different pipeline paths. This issue will be addressed as conditional task graphs.

\end{abstract}

\renewcommand{\abstractname}{Resumo}
\begin{abstract}
  resumo
\end{abstract}


\clearpage % Start a new page

\tableofcontents
\listoffigures
\listoftables

\chapter{Introduction}
In modern science, the need to analyze and process large amounts of data and information has been steadily but surely growing. This need has come to play a significant role to get relevant results from this data-sets.
\par These analyses are mostly done by computer applications that take data-sets and process raw data into useful information with scientific value allowing scientists in all the major fields to achieve discoveries with ease.

\subsection{Intro to HEP-Frame}
 Typically these applications are structured as a sequence of tasks in a pipeline of actions, where data can be modified at each pipeline stage, filtered out and/or output as a result. The next tasks in the pipeline do not process the elements that are filtered out. Actions often vary from intensive computing tasks to simple evaluations that may discard irrelevant data.
\par To aid the development of the software necessary for this kind of problems, a framework was created \textit{\textbf{HEP-Frame}}. This framework allows the user to create a pipeline of tasks and focusing on the result instead of writing the same boilerplate code. To achieve this, most of the duplicated code is generated at compile time through processors, and the repetitive tasks are automated through scripts. Alongside this with the intentions to make these calculations in the shortest time possible and the most efficient way possible. A special scheduler based on graph and list/queue schedulers was developed for this that allows efficient execution under multi-node computer clustering architectures, using parallel computing techniques.
This framework allows for a particular type of process that is incredibly useful for pipeline streaming that if at any stage a validation fails then the data-set is discarded and we can start working on the next set.





\subsection{Contexto scientific data analysis com pipelines que podem ter multiplas saidas que liga com o proximo ponto}
In some cases, an analysis may be carried out where more than one result can be a legitimate outcome for a specific validation. These different constraints can lead to different validations paths and can create possible outcome ramification. This type of task is commonly found in fields of science, such as particle analysis, where we can distinguish different types of particles depending on the values that specific tests obtain.
With the linking of tasks, we create a graph that describes a pipeline with a conditional flow.


\subsection{Definir mais em detalhe o que sao pipelines que podem ter multiplas saidas, que liga com o proximo ponto} 
This type of analysis can be described as a node connected to multiple exit nodes where we have one node for each possible successful outcomes and one to catch a possible fail case.
In each of these nodes, we can expect to have a computation operation and a condition for each possible outcome and respective node.
This can be thought out and modelled as a switch statement where a filter is applied and depending on the result raised by a condition, and a path is chosen for the next node. 

\subsection{Definir Mais em detalhe o que sao pipelines condicionais}
With the pipelines as mentioned earlier, we can describe the concept of conditional pipelines wherein each stage we can have multiple possible steps in which the next task to follow is calculated through a condition.

\subsection{Referir que ainda existe muito pouco trabalho na area de escalonadores para estas pipelines}
Although conditional pipelines have been applied to different domains, at this time, there is little work involved directly to scheduler.

\section{Challenges \& Goals}


%\subsection{ Definir com detalhe o objetivo da dissertacao}
%%The main goal of this work is to add new functionalities in the current HEP-Frame version, namely to integrate support in this framework for specification and execution of conditional pipelines using conditional task graphs. This allows HEP-Frame to support a broader range of pipelined applications while employing the existing optimization strategies for conventional task graphs, as well as implement new scheduling strategies for task pipelines that require conditional paths of execution.
The main goal of this work is to implement a scheduler in C++ that allows the efficient workload distribution and parallel execution of conditional pipelines. 
Most task and list schedulers use conventional execution and/or dependency graphs, where each node represents a task and the directed edges between nodes represent possible execution flows between tasks.
Conditional pipelines may not be possible to express as a traditional execution graph.
A single graph representation must include (i) dependencies among tasks, (ii) conditional paths of execution, and (iii) the filtering out of elements, preventing them from being executed by the remaining propositions in the pipeline.
An adequate conditional task graph representation must be designed, as it is critical to develop efficient strategies to schedule these tasks among workers.

This graph must support a set of pre-defined operations, such as validation if a given dataset element has been successfully processed by the pipeline, which may be occur without processing all propositions, while also being flexible to adapt to various optimisation techniques applied to the pipeline.
For instance, one common optimisation strategy at the hardware level that can be applied to this specific problem is speculative execution.
A preliminary analysis identified that branch prediction may be a vital component of the task scheduling strategy, as it allows subsequent conditional propositions to be processed simultaneously without knowing its result of the previous proposition yet.
This should be considered when defining the graph to ensure that, through its analysis, every execution flow with dependencies, conditional paths, and speculative execution provides the correct results.
Although there is little work conditional pipeline scheduling, traditional list schedulers and pipeline task schedulers, such as the one available in HEP-Frame, can be used as a starting point for this work.
The available implementations do not support conditional pipelines, so an initial adaptation of their scheduling strategies is required.
However, a series of steps are required to prepare and validate all the information required by a scheduler from an user implementation of a conditional pipeline.

%\subsubsection{Criar um escalonador com base em trabalho existente pra gerir execucao paralela das tarefas destas pipelines}

The conditional pipeline code must be analysed at compile time by a pre-processor to create its conditional task graph.
This graph must then be validated to ensure that the pipeline does not contain invalid dependencies among propositions and cycles tha may create deadlocks or infinite loops.
There are several algorithms, such as Depth First Search, Tarjan's strongly connected components algorithm, and Topological sorting, that can be used to achieve this purpose, but which is best for this problem must be first assessed.
Also, it is required to guarantee the validity of the multiple conditional propositions, which control the execution flow of the pipeline, through the their representation as a boolean formula.
An analysis of this formula, possibly using boolean satisfiability problem solver (also known as SAT solver), is crucial to identify if it has a solution, meaning that the pipeline is (or is not) executable.

Finally, the code definition of the conditional propositions, as well as their output, must be standardised to allow seamless coding and efficient scheduling of the pipeline execution.
The scheduler must expect a conditional proposition that has a set of pre-defined inputs, other than the input dataset of the application, and outputs, which are the identification of the next proposition to execute.

To assess the validity and efficiency of the proposed scheduler during its development an illustrative case study needs to be implemented.
This case study will then be replaced by a real application, related to the analysis of proton-proton collisions at the ATLAS Experiment, which is currently being developed by particle physicists at the Laboratório de Instrumentação e Física Experimental de Partículas.
After tuned and validated, the proposed scheduler should be integrated into HEP-Frame .




%With this, we also have some usage examples of this type of structure in other domains that can be helpful to understand common issues and how structuring and the representation at the computational level are done in a relatively compact and efficient way.
%This scheduler must allow executing these tasks with high levels of parallelism and data discarding that the current HEP-Frame schedule allows.
%Latter, the use of branch prediction and speculative execution, may be considered as an option being an optimisation used in conditional branching at the hardware level.

%\subsubsection{Garantir a correcao da pipeline}
%With the use of the pre-processor and the graph generated at compilation time that we discussed before, we can make some validations to the graph for any cycles that may create deadlocks of infinite loops. This type of validation can be done using Depth First Search, Tarjan's strongly connected components algorithm or Topological sorting (if we want that our graphs are acyclic, once that only DAGs can be topological sorted).
%Also, to guaranty the validity of the multiple Boolean conditional formulas, that will control all of the flow of the program, we are going to use the Boolean satisfiability problem solver also known as SAT solver. This type of plan allows us to input a reduced Boolean formula that the user submitted as a condition and tell us if that formula can determining if an interpretation that satisfies a given Boolean equation exists, that is if at any point this formula can be true.

%\subsubsection{Validar com casos de estudo (necessario escolher casos sinteticos e reais na fisica de particulas}

%An illustrative case study needs to be implemented to assess the performance and correctness of the scheduling of conditional task graphs in HEP-Frame. Once stable, the scheduler should be tested with a real case study, such as a particle physics scientific data analysis, since this type of applications may use conditional tasks in their pipelines. 

%\subsubsection{Integrar na HEP-Frame e objetivo dessa integracao}

%The developed scheduler should be accompanied by other tools to automatically parse a predefined signature of a task and identify its dependencies, identify data dependencies, validate the created graph, and measure several metrics useful for the scheduler. These tools should be efficient to either operate at compile- or run-time, depending on their purpose. Once validated in a contained code base, the scheduler and these tools should be integrated into HEP-Frame.

%\subsubsection{Definir uma estrutura em codigo para um utilizador criar tarefas e definir a sua pipeline}
%With the possible integration with the HEP-Frame we need to have a simple interface for the end user to work with, once that this framework is directed to non computer science personal.
%To provide this kind interface for the user we are using a pre-processor based approach where the user defines the tasks with return statements with the name of the tasks on a return statement in each conditional step. The pre-processor then takes all these statements, builds a graph that defines the control flow of the pipeline. With all the possible paths we can create a header file with an Enum that defines the integer equivalent to each task so that our program can be compiled. With this, we don't have to create a domain-specific language, and we can structure our code like a typical program. This method is incredibly flexible because it allows introducing more validation before compilation, like for example the graph validation that we are going to talk about new.

%******ESTES SAO DETALHES MUITO ESPECIFICOS NO PARAGRAFO ACIMA

%  *********pensei melhor e esta lista devia aparecer apenas no planeamento.

%In the end, we should plan for the following tasks:
%In the end, we should end up with the following components:
%\begin{itemize}
%    \item Implement a sample application with conditional tasks in C++ (1 month);
%    \item Design and implement a preprocessor to extract a graph representation of the tasks (1 month);
%    \item Design and implement a graph validator (1 month);
%    \item Design and implement a condition validator with sat-solvers (1 month);
%    \item Design and implement the task scheduler  (1 months);
%    \item Validate, profile, and tune the performance of the designed scheduler (1 months);
%    \item Integrate the scheduler in HEP-Frame and validate with a real case study (1 month).
%\end{itemize}



\section{Document Outline} 

%In this pre-dissertation report you will find the following chapters :
%\begin{itemize}
%    \item Introduction
%    \item Scheduling Conditional Tasks in HEP-Frame
%    \item Work Plan
%    \item Preliminary Work
%\end{itemize}

This chapter, \textit{Introduction}, presented the context of this work, its challenges and  the expected goals for this dissertation. 
The second chapter, \textit{Scheduling Conditional Tasks in HEP-Frame}, addresses the state of the art of schedulers, conditional pipelines and issues on efficient computing.
The third chapter, \textit{Work Plan}, details the required work to achieve the goals set for this project.
The last chapter, \textit{Preliminary Work}, describes the technical work already implemented, with some relevant results.
    
\chapter{Scheduling Conditional Tasks in HEP-Frame}

 *****Pequena intro a existencia de hardware cada vez mais complexo e ao software necessario para usar eficientemente ****

 *****Pipelines condicionais *em detalhe e com imagens (e dizer qu eas tarefas sao irregulares) ****


\section{Graphs to Represent Task Execution Order}

\subsection{Conditional Tasks}

\section{Allocation and Scheduling of Conditional Tasks}

\subsection{Computational Efficiency}


\section{HEP-Frame and Competitors}




 *****tudo o que precises ou que exista de carater geral desde boost threads, openMP, mpi, tbb, etc ****
 
 *****tudo o que procuraste (descrever o que e e o proposito, de escalonadores, list schedulers (e importante ter alguma coisa e provar que nao e util para o nosso problema, ferammentas DAG, etc, uteis para nos  ****
 
 *****Se calhar uma secao so para falar de grafos (algoritmos para construir e manipular so os que poderao ser uteis) ****

\subsection{Multicore}
\subsubsection{Descrever em detalhe todas as caracteristicas que sao importantes como multiplos cores, caches, vectorizacao, etc}
\subsubsection{Intel/AMD}

\subsection{Manycore}
\subsubsection{Descrever em detalhe todas as caracteristicas que sao importantes KNL, ver slides de AA}
\subsubsection{ARM}





\chapter{Work Plan}

The main goal of this work is to add new functionalities in the current HEP-Frame version, namely to integrate support in this framework for specification and execution of conditional task graphs. This allows HEP-Frame to support a wider range of pipelined applications, while employing the existing optimization strategies for conventional task graphs, as well as implement new scheduling strategies for task pipelines that require conditional paths of execution.
Most task schedulers use conventional execution graphs, where  the nodes represent tasks and the directed connections among them represent possible execution flows between tasks [1-9]. Conditional task graphs may not be possible to express as a traditional execution graph, as a single graph representation must include (i) dependencies among tasks, (ii) conditional paths of execution, and (iii) the filtering out of elements, preventing them from being executed by the remaining tasks. An adequate graph representation must be designed, as it is key to develop efficient strategies to schedule these tasks among workers.

\section{Proposed approach}
%\subsection{Ja detalhes concretos de como se deveria estruturar e implementar as coisas}



The developed scheduler should be accompanied by other tools to automatically parse a predefined signature of a task and identify its dependencies, identify data dependencies, validate the created graph, and measure several metrics useful for the scheduler. These tools should be efficient in order to either operate at compile- or run-time, depending on their purpose. Once validated in a contained code base, the scheduler and these tools should be integrated into HEP-Frame.
This way we can structure our project in the following components:
\begin{itemize}
    \item Preprocessor
    \item Graph Creator
    \item Graph validator
    \item Conditional Validator
    \item Scheduler
\end{itemize}

\begin{figure}[h]
\includegraphics[width=15cm]{img/Tools-pipeline}
\end{figure}

This way the preprocessor, graph creator and graph validation run before compilation with a pipeline like behaviour.
This process will  occur before compilation and its end result will end up being an header file with an enum that attributes an integer to each of the user defined functions and a separated file with the graph definition to be used in the scheduler.


%\subsection{Definicao do prototipo das tarefas (restricoes e codigo}

%****como se definiu a sintaxe (ex, porque nao usar vars globais)
%****explicar o mecanismo de CRIAR e PASSAR de Prop2 para um valor qe o escalonador entenda

This scheduler will execute the functions defined by the user that have the following interface 
\begin{lstlisting}
#include "props.h"

int prop1(int id, Data data) { 
    if(...) 
        return Prop2; 
    if(...) 
        return Prop3; 
    if(...) 
        return FAIL; 
}
\end{lstlisting}
with this interface we can pass a thread id for the scheduler and the data field for the dataset that is being executed right now. Together with the header file we can distinguish the user defined functions from the rest of the code.
With all the pre processor steps finished and all the validations a prop header file is created to allows us to define a Enum interface for all the defined function so to get a final function equivalent to this:

\begin{lstlisting}
int Prop1(....) { 
    if(...) 
        return 1; 
    if(...) 
        return 2; 
    if(...) 
        return -1; 
} 
\end{lstlisting}{}
that will be able to actually able to compile.




%\subsection{qual o pre processamento e como o fazer concretamente (tudo desde criar Dag, se as dependencias sao validas,etc}
The preprocessor will be a separate program that parses the source files written by the user and by the context given witha the function definitons and the imported source files we can uderstand what are the functions belonging to our pipeline. After parsing this functions we create a file that describes the inter connections of all the tasks, wich will form a graph.
After this the file will pass a validation process, wich will detect any possible cycle with topological sorting, (Also Known as Dependency resolution) if we want to have acyclic graph and need to order the tasks that constitute them then this seems the best option, because it can check both properties at the same time.
With the graph validated we then need to validate the condition that decide if a certain branch of the branch is followed. If a condition is invalid then we cant compile because that task will never be reached.
With all the validations we can them procede to generate a header file with a Enum that associates a function to a integer.

%%\subsection{qual a abordagem a tomar para o escalonador}


%%***ideias, imagens, como devera ser dividida a carga***



%\subsubsection{branch prediction}
One common optimisation strategy used at the hardware level to improve the performance of conditional branches is branch prediction. It was already identified in a preliminary analysis of the problem that branch prediction may be a key component of the task scheduling strategy, as it allows subsequent tasks to be processed simultaneously with a current task without knowing its result yet. This should be considered when defining the graph to ensure that, through the analysis of this graph, every code execution with task dependencies, conditional paths, and branch prediction provides the correct results.


%\subsubsection{definicao ja dos requisitos que o benchmark sintetico tera}
%\subsubsection{definicao ja dos requisitos que o benchmark sintetico tera}
An illustrative case study needs to be implemented to assess the performance and correctness of the scheduling of conditional task graphs in HEP-Frame. Once stable, the scheduler should be tested with a real case study, such as a particle physics scientific data analysis, since this type of applications may use conditional tasks in their pipelines. 

\section{Work Timeline}

The timeline of the proposed work is as follows:
\begin{itemize}
    \item Implement a sample application with conditional tasks in C++ (1 month);
    \item Design and implement a preprocessor to extract a graph representation of the tasks (1 month);
    \item Design and implement a graph validator (1 month);
    \item Design and implement a condition validator with sat-solvers (1 month);
    \item Design and implement the proposed scheduler  (1 months);
    \item Validate, profile, and tune the performance of the scheduler (1 months);
    \item Integrate the scheduler in HEP-Frame and validate it with a real case study (1 month);
    \item Write the dissertation (1 month).
\end{itemize}
\chapter{Preliminary work}
Some work was developed during this initial stage of the thesis work, namely on the research and implementation of different graphs and strategies to validate them.
This chapter includes a explanation for the type of graphs that may be used, alongside their key characteristics, concurrency models and the different components mentioned in the previous chapters.

\section{Graph Theory}

Graph theory classifies different graphs by their properties and characteristics. Some of the more relevant types of graphs for this dissertation are: 
\begin{itemize}
    \item Complete graph: every pair of distinct vertices is connected by a unique edge;
    \item Directed graph: where edges have a direction associated with them;
    \item Acyclic graph: a finite directed graph with no directed cycles;
    \item Strongly connected graph: when every vertex is reachable from every other vertex;
    \item Supergraph:  formed by adding vertices, edges, or both to a given graph;
    \item Component (or connected component): vertices connected to each other by paths and which is connected to no additional vertices in the supergraph;
    \item Subgraph another graph formed from a subset of the vertices and edges;
    \item Dependency graph representation dependencies of several objects towards each other; if a dependency graph does not have any circular dependencies, it forms a Direct Acyclic graph;
    \item Task graph tasks are vertices with edges between them representing dependencies between tasks.
\end{itemize}

As referred in the previous chapters, these graphs are required to represent the task execution-order (addressed in this work as TEG).
% was understood in the previous chapters that we need graphs to represent the task execution-order -- \textbf{TEG}.

A suitable TEG graph needs to respect the following properties: 
\begin{enumerate}
    \item Must represent tasks and paths connecting tasks (vertices and edges), where the edges have a direction associated with them (directed graph);
    \item Must have one start vertex(task) and always reach one end vertex(task);
    \item Must represent the weight of a task;
    \item Must represent dependencies between tasks;
    \item Each task may be executed at most once (acyclic graph);
    \item May have an hierarchical representation with subgraphs;
    \item Subgraphs may be strongly connected and complete graph;
    \item Must support multiple paths between the starting and the ending   vertices.
\end{enumerate}{}


Following these criteria the previous graph types can be classified as \textit{suitable}, \textit{maybe} or \textit{not suitable} for TEG, as shown in table \ref{tab:teg}.

\begin{table}[tbh!]
\begin{center}
\begin{tabular}{l|l}
Complete graph                     & \ding{55}    \\ \hline
Directed  graph                    & \ding{51} \\ \hline
Acyclic graph                      & ?         \\ \hline
Strongly connected graph           & ?         \\ \hline
Supergraph                         & \ding{55} \\ \hline
Component (or connected componect) & \ding{55} \\ \hline
Subgraph                           & \ding{51} \\ \hline
Dependency graph                   & \ding{51}\\ \hline
Task graph                         & \ding{51}     
\end{tabular}
\end{center}
\caption{Suitability of the representation of a task execution-order in different graphs.}
\label{tab:teg}
\end{table}


\section{Pre-processor}

\par A preprocessor  capable of parsing a C/C++ source file, analyse it and generate the corresponding conditional graph has already started development. 
This is a simple parser written in \textit{python} using this language builtin \textit{regular expression} engine. With the use of regular expressions the parser can understand when it is inside of a function, using a reference counter of the bracket symbol, and increasing or decreasing this counter when a \textit{{} or a \textit{}} is found. Together with this we can also find out if a given function belongs to the user pipeline analysing the function return type and signature.
The return values of each function can be identified and considered the next step of the pipeline, thus creating the graph based on them. It is also required to create a syntactic fail node for when a stage in the pipeline fails and we want to halt all the computation for this data-set and pass the execution to the next in line, allowing for the most efficient execution possible.

After creating the graph it is needed to save its state to disk for future loading, validation of correctness and as a intermediate structure that the scheduler is going to use to create a TEG. 
Different formats and ways to encode and save this graph were explored, including the language binary representation of the \textit{struct}, generic binary representation, domain-specific language and generic text formats. All these formats have trade-offs in speed and readability, but this being a dissertation as a goal to extend a framework that is supposed to be user-friendly, it was opted to go with the more readable and easier to analyse format, such as generic text formats. Within this type of format, it was concluded that the best options were JSON, YAML and XML, for they provide the better parser libraries and visualisation/editing tools. 
After some preliminary testing, XML was excluded for being too verbose and therefore infringing the requirements defined in the previous stage. JSON can be used as a YAML subset; therefore, it can be used in tools designed for booth formats. For all these reasons it was concluded that JSON is the best format to be used to encode our graph in a file that is going to be saved to disk and later used in some additional tasks.

\section{Graph validation}

Deadlocks and infinite loops can occur in a conditional pipeline, as the user has a high degree of freedom when defining it.

\begin{figure}[h]
\begin{center}
\includegraphics[width=10cm]{latex/img/CiclicGraph}
\end{center}
\caption{A tree-like graph representing possible execution flows among conditional propositions with a cycle (red arrow).}
\label{fig:graph}
\end{figure}

To prevent such cases an algorithm to analyse the graph to assess if a graph is acyclic should be used.
For instance, figure \ref{fig:graph} presents a graph that has a cycle that must be avoided.
For this the Depth First Search traversal, Tarjan's strongly connected components algorithm or Topological sorting algorithms can be used.

\section{Condition validation}
The graph should be able to contain conditional tasks to control the flow through the graph.
User coded conditions of the propositions must be validated to check if the condition is satisfiable.
In the case it is not satisfiable the proposition can remove or out right refuse to run the program and warn the user to change it.
The conditional pipeline can be considered as boolean satisfiable problem and a SAT solver should be used to validate these conditions.
The Boolean satisfiability problem also abbreviated to SATISFIABILITY or SAT is the problem of determining if a interpretation that satisfies a given boolean formula exists. 
SAT solver is the program that given a first order boolean formula (TRUE, FALSE, OR, AND, NOT)  solves this type of problems.
Boolean conditions that are not first order must be reduce to first order. For instance, if a condition (x>0) AND (A OR B) can be transformed in to C AND (A OR B). 


\section{Speculative execution/ concurrency model}
One common optimisation strategy used at the hardware level to improve the performance of conditional branches is branch prediction and speculative execution. It was already identified branch prediction may be a key component of the task scheduling strategy, as it allows subsequent tasks to be processed simultaneously with a current task without knowing its result yet. This should be considered when defining the graph to ensure that, through the analysis of this graph, every code execution with task dependencies, conditional paths, and branch prediction provides the correct results.
With this idea in mind it was tried to search for a concurrency model that could easily satisfy this kind of problem and end up finding for multithreading Posix threads, boost threads and the  actors model and for multi-processing async/await, MPI and CSP.
Probably most of the multithreading implementation will resort to Boost threads and/or Posix threads, but some of the speculative execution may be done through the use of the async/await paradigm.

\section{Synthetic test case}
A preliminary synthetic data analyses pipeline was developed in this phase, which emulates the type of conditional propositions that could benefit from the proposed schedular. This benchmark allows to abstract the development from more complex applications, making it easier and simple to implement and test new ideas that may come by as a possible way to speed up the user source code.

This type of program is characterized by irregular workload, which means that tasks have variable execution times within the same pipeline. These tasks can range from simple verification of integers to complex and/or compute intensive operations, such as matrix-matrix operations. To recreate correctly the workload this irregular behaviour must be replicated.
Using simulated data and computations, opposed to random process sleeps commonly using in synthetic scheduler benchmarking, will also help to get a better feel of real world performance for the proposed scheduler strategies. This will allow the creation of a standard group of tests from the most normal workload to the extreme edge cases to prove the robustness of this work.

To create a dataset that can be used a synthetic data block was created with the following structure:
\begin{lstlisting}
  std::vector<float> m_vector_float_a;
  std::vector<float> m_vector_float_b;
  std::vector<float> m_vector_float_c;

  std::vector<int> m_vector_int_a;
  std::vector<int> m_vector_int_b;
  std::vector<int> m_vector_int_c;

  std::vector<std::vector<int>> m_matrix_a;
  std::vector<std::vector<int>> m_matrix_b;
  std::vector<std::vector<int>> m_matrix_c;

  std::vector<std::vector<float>> m_matrix_float_a;
  std::vector<std::vector<float>> m_matrix_float_b;
  std::vector<std::vector<float>> m_matrix_float_c;

  int m_int_a{};
  int m_int_b{};
  int m_int_c{};

  float m_float_a;
  float m_float_b;
  float m_float_c;

  std::vector<int> m_list_ints;

  std::vector<float> m_list_floats;

\end{lstlisting}
A group of vectorial, single arithmetic operation and matrix-matrix operations can be described as:
\begin{lstlisting}
template <typename T>
void prod_matrix(std::vector<std::vector<T>> &a, std::vector<std::vector<T>> &b,
                 std::vector<std::vector<T>> &c) ;

template <typename T>
void dot_prod_matrix(std::vector<std::vector<T>> &a,
                     std::vector<std::vector<T>> &b,
                     std::vector<std::vector<T>> &c) ;

template <typename T>
bool is_positive(T number) ;

template <typename T>
void add(T &number) ;

template <typename T>
void filter_list(std::vector<T> &list, T number) ;
\end{lstlisting}

Using the C++ template system it is possible to quickly develop multiple generic functions over our Data structure. This way building a pipeline for proof-of-concept purposes can be done quickly and easily with most of the code being generate by the compiler.


\end{document}
