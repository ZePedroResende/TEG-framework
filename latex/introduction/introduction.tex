\chapter{Introduction}


\section{ Context}


\subsection{Scientific data analysis}
\subsection{Intro to HEP-Frame}
\subsection{Contexto scientific data analysis com pipelines que podem ter multiplas saidas que liga com o proximo ponto}
\subsection{Definir mais em detalhe o que sao pipelines que podem ter multiplas saidas, que liga com o proximo ponto}
\subsection{Definir mais em detalhe o que sao pipelines condicionais}
\subsection{Referir que ainda existe muito pouco trabalho na area de escalonadores para estas pipelines}




\section{challenges \& goals}


\subsection{ Definir com detalhe o objetivo da dissertacao}
\subsubsection{Definir uma estrutura em codigo para um utilizador criar tarefas e definir a sua pipeline}
\subsubsection{Criar um escalonador com base em trabalho existente pra gerir execucao paralela das tarefas destas pipelines}
\subsubsection{Garantir a correcao da pipeline}
\subsubsection{Definir grafos e indicar o seu proposito neste problema}
\subsubsection{Validar com casos de estudo (necessario escolher casos sinteticos e reais na fisica de particulas}
\subsubsection{Integrar na HEP-Frame e objetivo dessa integracao}
\subsubsection{Uma lista de tarefas a realizar que sao basicamente os pontos anteriores mas resumidos}




\section{Estrutura documento}