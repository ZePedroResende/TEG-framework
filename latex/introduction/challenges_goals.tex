
\section{Challenges \& Goals}


%\subsection{ Definir com detalhe o objetivo da dissertacao}
%%The main goal of this work is to add new functionalities in the current HEP-Frame version, namely to integrate support in this framework for specification and execution of conditional pipelines using conditional task graphs. This allows HEP-Frame to support a broader range of pipelined applications while employing the existing optimization strategies for conventional task graphs, as well as implement new scheduling strategies for task pipelines that require conditional paths of execution.
The main goal of this work is to implement a scheduler in C++ that allows the efficient workload distribution and parallel execution of conditional pipelines. 
Most task and list schedulers use conventional execution and/or dependency graphs, where each node represents a task and the directed edges between nodes represent possible execution flows between tasks.
Conditional pipelines may not be possible to express as a traditional execution graph.
A single graph representation must include (i) dependencies among tasks, (ii) conditional paths of execution, and (iii) the filtering out of elements, preventing them from being executed by the remaining propositions in the pipeline.
An adequate conditional task graph representation must be designed, as it is critical to develop efficient strategies to schedule these tasks among workers.

This graph must support a set of pre-defined operations, such as validation if a given dataset element has been successfully processed by the pipeline, which may be occur without processing all propositions, while also being flexible to adapt to various optimisation techniques applied to the pipeline.
For instance, one common optimisation strategy at the hardware level that can be applied to this specific problem is speculative execution.
A preliminary analysis identified that branch prediction may be a vital component of the task scheduling strategy, as it allows subsequent conditional propositions to be processed simultaneously without knowing its result of the previous proposition yet.
This should be considered when defining the graph to ensure that, through its analysis, every execution flow with dependencies, conditional paths, and speculative execution provides the correct results.
Although there is little work conditional pipeline scheduling, traditional list schedulers and pipeline task schedulers, such as the one available in HEP-Frame, can be used as a starting point for this work.
The available implementations do not support conditional pipelines, so an initial adaptation of their scheduling strategies is required.
However, a series of steps are required to prepare and validate all the information required by a scheduler from an user implementation of a conditional pipeline.

%\subsubsection{Criar um escalonador com base em trabalho existente pra gerir execucao paralela das tarefas destas pipelines}

The conditional pipeline code must be analysed at compile time by a pre-processor to create its conditional task graph.
This graph must then be validated to ensure that the pipeline does not contain invalid dependencies among propositions and cycles tha may create deadlocks or infinite loops.
There are several algorithms, such as Depth First Search, Tarjan's strongly connected components algorithm, and Topological sorting, that can be used to achieve this purpose, but which is best for this problem must be first assessed.
Also, it is required to guarantee the validity of the multiple conditional propositions, which control the execution flow of the pipeline, through the their representation as a boolean formula.
An analysis of this formula, possibly using boolean satisfiability problem solver (also known as SAT solver), is crucial to identify if it has a solution, meaning that the pipeline is (or is not) executable.

Finally, the code definition of the conditional propositions, as well as their output, must be standardised to allow seamless coding and efficient scheduling of the pipeline execution.
The scheduler must expect a conditional proposition that has a set of pre-defined inputs, other than the input dataset of the application, and outputs, which are the identification of the next proposition to execute.

To assess the validity and efficiency of the proposed scheduler during its development an illustrative case study needs to be implemented.
This case study will then be replaced by a real application, related to the analysis of proton-proton collisions at the ATLAS Experiment, which is currently being developed by particle physicists at the Laboratório de Instrumentação e Física Experimental de Partículas.
After tuned and validated, the proposed scheduler should be integrated into HEP-Frame .




%With this, we also have some usage examples of this type of structure in other domains that can be helpful to understand common issues and how structuring and the representation at the computational level are done in a relatively compact and efficient way.
%This scheduler must allow executing these tasks with high levels of parallelism and data discarding that the current HEP-Frame schedule allows.
%Latter, the use of branch prediction and speculative execution, may be considered as an option being an optimisation used in conditional branching at the hardware level.

%\subsubsection{Garantir a correcao da pipeline}
%With the use of the pre-processor and the graph generated at compilation time that we discussed before, we can make some validations to the graph for any cycles that may create deadlocks of infinite loops. This type of validation can be done using Depth First Search, Tarjan's strongly connected components algorithm or Topological sorting (if we want that our graphs are acyclic, once that only DAGs can be topological sorted).
%Also, to guaranty the validity of the multiple Boolean conditional formulas, that will control all of the flow of the program, we are going to use the Boolean satisfiability problem solver also known as SAT solver. This type of plan allows us to input a reduced Boolean formula that the user submitted as a condition and tell us if that formula can determining if an interpretation that satisfies a given Boolean equation exists, that is if at any point this formula can be true.

%\subsubsection{Validar com casos de estudo (necessario escolher casos sinteticos e reais na fisica de particulas}

%An illustrative case study needs to be implemented to assess the performance and correctness of the scheduling of conditional task graphs in HEP-Frame. Once stable, the scheduler should be tested with a real case study, such as a particle physics scientific data analysis, since this type of applications may use conditional tasks in their pipelines. 

%\subsubsection{Integrar na HEP-Frame e objetivo dessa integracao}

%The developed scheduler should be accompanied by other tools to automatically parse a predefined signature of a task and identify its dependencies, identify data dependencies, validate the created graph, and measure several metrics useful for the scheduler. These tools should be efficient to either operate at compile- or run-time, depending on their purpose. Once validated in a contained code base, the scheduler and these tools should be integrated into HEP-Frame.

%\subsubsection{Definir uma estrutura em codigo para um utilizador criar tarefas e definir a sua pipeline}
%With the possible integration with the HEP-Frame we need to have a simple interface for the end user to work with, once that this framework is directed to non computer science personal.
%To provide this kind interface for the user we are using a pre-processor based approach where the user defines the tasks with return statements with the name of the tasks on a return statement in each conditional step. The pre-processor then takes all these statements, builds a graph that defines the control flow of the pipeline. With all the possible paths we can create a header file with an Enum that defines the integer equivalent to each task so that our program can be compiled. With this, we don't have to create a domain-specific language, and we can structure our code like a typical program. This method is incredibly flexible because it allows introducing more validation before compilation, like for example the graph validation that we are going to talk about new.

%******ESTES SAO DETALHES MUITO ESPECIFICOS NO PARAGRAFO ACIMA

%  *********pensei melhor e esta lista devia aparecer apenas no planeamento.

%In the end, we should plan for the following tasks:
%In the end, we should end up with the following components:
%\begin{itemize}
%    \item Implement a sample application with conditional tasks in C++ (1 month);
%    \item Design and implement a preprocessor to extract a graph representation of the tasks (1 month);
%    \item Design and implement a graph validator (1 month);
%    \item Design and implement a condition validator with sat-solvers (1 month);
%    \item Design and implement the task scheduler  (1 months);
%    \item Validate, profile, and tune the performance of the designed scheduler (1 months);
%    \item Integrate the scheduler in HEP-Frame and validate with a real case study (1 month).
%\end{itemize}
