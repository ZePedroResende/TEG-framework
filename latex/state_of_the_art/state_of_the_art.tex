\chapter{Scheduling Conditional Tasks in HEP-Frame}

 *****Pequena intro a existencia de hardware cada vez mais complexo e ao software necessario para usar eficientemente ****

 *****Pipelines condicionais *em detalhe e com imagens (e dizer qu eas tarefas sao irregulares) ****


\section{Graphs to Represent Task Execution Order}

\subsection{Conditional Tasks}

\section{Allocation and Scheduling of Conditional Tasks}

\subsection{Computational Efficiency}


\section{HEP-Frame and Competitors}




 *****tudo o que precises ou que exista de carater geral desde boost threads, openMP, mpi, tbb, etc ****
 
 *****tudo o que procuraste (descrever o que e e o proposito, de escalonadores, list schedulers (e importante ter alguma coisa e provar que nao e util para o nosso problema, ferammentas DAG, etc, uteis para nos  ****
 
 *****Se calhar uma secao so para falar de grafos (algoritmos para construir e manipular so os que poderao ser uteis) ****

\subsection{Multicore}
\subsubsection{Descrever em detalhe todas as caracteristicas que sao importantes como multiplos cores, caches, vectorizacao, etc}
\subsubsection{Intel/AMD}

\subsection{Manycore}
\subsubsection{Descrever em detalhe todas as caracteristicas que sao importantes KNL, ver slides de AA}
\subsubsection{ARM}




